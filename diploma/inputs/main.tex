% TODO: We need more water here.

\section{Система управления персонажем}

Работа системы управления персонажем состоит из двух стадий: оптимизации и прямой динамики, повторяемых в цикле. Во время оптимизации вычисляются силы и моментов сил, которые должны должны сгенерировать приводы, и силы реакции опоры. Результаты оптимизации передаются в алгоритм прямой динамики, который находит ускорения для дальнейшего интегрирования.

Внешние возмущения, если пресутсвуют, также передаются в алгоритм прямой динамики. Ошибки, которые они вносят в движение, компенсируются оптимизатором на следующих циклах.

На рисунке \ref{fig:architecture} схематично изображена работа системы.

\begin{figure}[h]
  \begin{minipage}{\textwidth}
    \centering
    \begin{tikzpicture}
        \node [rectangle, minimum width = 4cm, minimum height = 2cm, draw] at (-3, 0) {Оптимизация};
        \node [rectangle, minimum width = 4cm, minimum height = 2cm, draw] at (3, 0) {Прямая динамика};
        \draw [-{Latex[length=3mm, width=2mm]}] (-1,0)  -- node[above = 1mm] {$u, f$} (1,0);
        \draw [-{Latex[length=3mm, width=2mm]}] (3, 2.5) -- node[right = 1mm] {$f_{d}$} (3, 1);
        \draw [-] (3, -1) -- (3, -2.5);
        \draw [-] (3, -2.5) -- node[below = 1mm] {$q, \dot{q}, \ddot{q}$} (-3, -2.5);
        \draw [-{Latex[length=3mm, width=2mm]}] (-3, -2.5) -- (-3, -1);
      \end{tikzpicture}
    \caption{Работа системы управления}
    \label{fig:architecture}
  \end{minipage}
\end{figure}

Отметим, что похожая схема работы описана в \cite{AbeSP}. Однако она имеет другую формулировку задачи оптимизации, что является важным отличием, делающим систему, предложенную в данной работе, более надежной.

% В этот кусок текста тоже нужно залить воды.
% NOTE: Решил что вот тут напишем про это
% когда все идет по пизде
% Применительно к анимациям в видео играх это может быть полезно.
% Отличительной особенностью данного подхода является формулировка задачи оптизации
% которая позволяет нарушить физическую корректность, чтобы система управления
% Применительно к анимациям в видео играх это может быть разумный выбор.

% TODO: Advantages of optimization

\subsection{Оптимизация}

% At least two

% TODO: as mentioned above we use optimization to embed several motion objectives in out system.

% TODO: Each of them will be formulated as follows
% - set desired acceleration
% - formulate function which minimization will bring values to desired

% Формулирование целей

Задача оптимизации имеет следующий вид
\begin{align*}
  \min_{\ddot{q}, u, f} &\ \omega_{t} h_{t}(\ddot{q}) + \omega_{c} h_{c}(\ddot{q}) + \omega_{p} h_{p}(\ddot{q}) + \omega u^{T} \begin{bmatrix} I_{6} & O \\ O & O \end{bmatrix} u \\
  s.t. &\ H \ddot{q} + C(q, \dot{q}) + G(q) = u + J^{T} f \tag{2.1}\label{eq:2.1} \\
       &\ J_{sup} \ddot{q} + \dot{J_{sup}} \dot{q} = 0 \tag{2.2}\label{eq:2.2} \\
       &\ 0 \le n(f) \tag{2.3}\label{ineq:2.3} \\
       &\ \tau(f) \le \mu n(f) \tag{2.4}\label{ineq:2.4}
\end{align*}

Целевая функция оптимизации состоит из четырех слагаемых. Первые три из них -- $h_{t}$, $h_{c}$ и $h_{p}$ -- отвечают за следование опорной анимации, за положения центра масс и за положения центра давления соответственно. Последнее стремиться минимизировать силы и моменты сил, генерируемые в корневом шарнире. Веса $w_{t}$, $w_{c}$, $w_{p}$ и $w$ -- позволяют регулировать значимость вклада каждого из слагаемых.

Уравнение \ref{eq:2.1} согласует результаты оптимизации с уравнением динамики кинематического дерева. Уравнение \ref{eq:2.2} оставляет точки контакта с поверхностью неподвижными. Неравенства \ref{ineq:2.3} и \ref{ineq:2.4} отвечают за направление силы реакции опоры.

Функции $h_{t}$, $h_{c}$ и $h_{p}$ формулируются так, чтобы получившаяся задача оптимизации была квадратичной. Такое ограничение дает предсказуемое время работы оптимизатора и стабильные результаты.

\subsubsection{Следование за опорной анимацией}

Функция $h_{t}$ должна как можно точнее приближать получаемое движение к опорной анимации. На первый взгляд, для этого, достаточно определить $h_{t}$ таким образом, чтобы в результате оптимизации ускорение совпадало с тем, что взято из опорной анимации. Однако, при таком определении, в случае отклонения получаемого движения от опорной анимации оно не будет скорректировано обратно. Поэтому $h_{t}$ будет определена следующим образом
\begin{equation*}
  h_{t}(\ddot{q}) = \lVert \ddot{q} + s_{t} (q - q_{ref}) + d_{t} (\dot{q} - \dot{q}_{ref}) - \ddot{q}_{ref} \rVert_{2}^{2},
\end{equation*}
где $s_{t}$ и $d_{t}$ -- коэффициенты, а $q_{ref}$, $\dot{q}_{ref}$ и $\ddot{q}_{ref}$ -- это положение, скорость и ускорение взятые из опорной анимации.

Выражение $s_{t} (q - q_{ref}) + d_{t} (\dot{q} - \dot{q}_{ref})$ использованное в определении $h_{t}$ называется \break пропорционально-дифференцирующий регулятор, а коэффициенты $s_{t}$ и $d_{t}$ -- пропорциональный и дифференциальный соответственно. Использование этого регулятора позволяет преодолеть проблему, описанную выше, а также смягчить ошибки связанные с работой с числами с плавающей точкой.

% Отметим, что по аналогичным причинам пропорционально-дифференцирующий регулятор будет использован при определении остальных функций.

% TODO: контроль

\subsubsection{Управление положением центра масс}

Функция $h_{c}$ должна поддерживать проекцию положения центра масс на поверхность внутри выпуклой оболочки, образованной точками контакта с поверхностью, поскольку в противному случае персонаж перевернется.

Для этого $h_{c}$ определяется таким образом, чтобы в результате оптимизации текущее положение центра масс двигалось в направлении, предварительно выбранного, опорного положения. А точнее, ускорение центра масс принимало следующий вид
\begin{equation*}
  \ddot{c}_{des} = - s_{c} (c - c_{ref}) - d_{c} \dot{c}, \tag{2.6}\label{eq:2.6}
\end{equation*}
где $s_{t}$ и $d_{t}$ -- коэффициенты, а $c_{ref}$ -- опорное положение.

Умножением обоих частей на массу, уравнение \ref{eq:2.6} принимает вид
\begin{equation*}
  m \ddot{c}_{des} = - s_{c} m (c - c_{ref}) - d_{c} m \dot{c}. \tag{2.7}\label{eq:2.7}
\end{equation*}

Заменой $m \ddot{c}_{des}$ и $m \dot{c}$ через уравнения \ref{eq:1.10} и \ref{eq:1.11}, уравнение \ref{eq:2.7} принимает вид
\begin{equation*}
  \dot{P}_{des} = - s_{c} m (c - c_{ref}) - d_{c} P. \tag{2.8}\label{eq:2.8}
\end{equation*}

Таким образом управление положением центра масс сводится к управлению величиной импульса, что сделать не сложно, поскольку есть зависимость между ним и обобщенными координатами, скоростями и ускорениями, выраженная в виде уравнения \ref{eq:1.5}. В итоге $h_{c}$ определяется следующим образом
\begin{equation*}
  h_{c}(\ddot{q}) = \lVert P - P_{des} \rVert_{2}^{2} = \lVert A_{P}\ddot{q} - \dot{A_{P}} \dot{q} - P_{des} \rVert_{2}^{2}.
\end{equation*}

\subsubsection{Управление положением центра давления}

% Вот тут начнется прикол

\subsubsection{Положением точек контакта с поверхностью}

% поскольку мы занимаемся статическим балансам необходимо следить за тем чтобы опорные точки не отрывались от поверхности на которой стоит ля

\subsubsection{Направление силы реации опры}

\subsection{Прямая динамика}

% Тут еще надо сказать про correct feedback error
% Использование прямой динамики необходимо по двум причинам
% 2. мы не знаем внешние возмущения in advance, поэтому не можем включить их в optimizer

Для приведения кинематического дерева в движение необходимо использование алгоритма прямой динамики. В данной работе используется алгоритм articulated rigid body, разработанный и описанный в \cite{Featherstone}, который имеет меньшую вычислительную сложностью в сравнении с остальными алгоритмами прямой динамики.


% Как упоминалось в предыдущем разделе, цикл обратной/прямой динамики выполняется в первую очередь для того, чтобы учесть внешние возмущения в системе.
% Алгоритм оптимизации не знает внешних сил. Вместо этого он полагается на обратную связь по ошибкам отслеживания для коррекции возмущений с помощью PD-контроллера. Включив цикл прямой динамики, можно точно смоделировать динамику удара.
% Используемый алгоритм прямой динамики — Featherstone. Featherstone — это эффективный алгоритм с уменьшенными координатами O(n), который решает уравнения движения посредством рекурсии. В отличие от конкурирующего метода составного твердого тела O(n3), метод Физерстоуна работает быстрее, когда n > 9 [11].

% результирующая сил трения и сил нормальной реакции опоры не может иметь произвольное направление, поскольку сила трения и N  через коэффициент трения

% Уравнение прямой динамики

% \subsection{Реализация системы}
% Для реализации системы использовались pinocchio и cvxopt


% Оптимизация имеет четыре ограничения. Первое из них согласует результаты с уравнением динамики кинематического дерева. Второе оставляет точки контакта с поверхностью неподвижными. Последние два отвечают за направление 

% Ограничения оптимизации наложены для получения физически корректных результатов. Первое из них согласует результаты оптимизации с уравнением динамики кинематического дерева.

% Первое .
% Второе отвечает за направление силы нормальной реакции опоры.
% 

% Ограничения оптимизации

% Первое из них -- это уравнение динамики. Второе 

% Функции  -- отвечают за следование опорной анимации, за контроль положения центра масс и за контроль положения центра давления. Веса $w_{t}$, $w_{c}$ и $w_{p}$ позволяют настраивать вклад каждой из них. Последнее слагаемое целевой функции оптимизации минимизирует силы и моменты сил, генерируемые в корневом шарнире.

% Ограничения оптимизации 

% делая контролеер более

% Второе из ограничений

% Целевые функции отвечают за достижение целей контроллера.

% Ограничение оптимизации отвечают за сохранение физической

% Первое ограничение требует выполнения уравнения

% уравнением движения, коэффи

