\section{Система управления персонажем}

Работа системы управления персонажем состоит из двух стадий: оптимизации и прямой динамики, повторяемых в цикле. Во время оптимизации вычисляются силы и моментов сил, которые должны должны сгенерировать приводы, и силы реакции опоры. Результаты оптимизации передаются в алгоритм прямой динамики, который находит ускорения, для дальнейшего интегрирования.

Внешние возмущения, если пресутсвуют, также передаются в алгоритм прямой динамики. Ошибки, которые они вносят в движение, компенсируются оптимизатором на следующих циклах.

На рисунке \ref{fig:architecture} схематично изображена работа системы.

\begin{figure}[h]
  \begin{minipage}{\textwidth}
    \centering
    \begin{tikzpicture}
        \node [rectangle, minimum width = 4cm, minimum height = 2cm, draw] at (-3, 0) {Оптимизация};
        \node [rectangle, minimum width = 4cm, minimum height = 2cm, draw] at (3, 0) {Прямая динамика};
        \draw [-{Latex[length=3mm, width=2mm]}] (-1,0)  -- node[above = 1mm] {$u, f$} (1,0);
        \draw [-{Latex[length=3mm, width=2mm]}] (3, 2.5) -- node[right = 1mm] {$f_{d}$} (3, 1);
        \draw [-] (3, -1) -- (3, -2.5);
        \draw [-] (3, -2.5) -- node[below = 1mm] {$q, \dot{q}, \ddot{q}$} (-3, -2.5);
        \draw [-{Latex[length=3mm, width=2mm]}] (-3, -2.5) -- (-3, -1);
      \end{tikzpicture}
    \caption{Работа системы управления}
    \label{fig:architecture}
  \end{minipage}
\end{figure}

Отметим, что похожая схема работы описана в \cite{AbeSP}. Однако она имеет другую формулировку задачи оптимизации, что является важным отличием, делающим, систему, предложенную в данной работе, более надежной.

\subsection{Оптимизация}

% Оптимизатор решает задачу квадратичного программирования при линейных условия следующего вида

\subsubsection{Следование за опорной анимацией}

% Тут тоже очев

\subsubsection{Контроль положения центра масс}

% Ну тут очев

\subsubsection{Контроль положения центра давления}

% Вот тут начнется прикол

\subsubsection{Контроль положения опорных точек}

% поскольку мы занимаемся статическим балансам необходимо следить за тем чтобы опорные точки не отрывались от поверхности на которой стоит ля

\subsubsection{Трение}

% результирующая сил трения и сил нормальной реакции опоры не может иметь произвольное направление, поскольку сила трения и N  через коэффициент трения



% \subsection{Реализация системы}
% Для реализации системы использовались pinocchio и cvxopt