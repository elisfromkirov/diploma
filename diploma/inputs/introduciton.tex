\section{Введение}

Одна из основных задач, стоящих перед разработчиками и исследователями в области анимации, -- это воспроизведение разнообразия движений человека, с целью повышения реализма персонажей и их взаимодействий с игровым миром. Несмотря на сложности, связанные с недостаточным пониманием всех тонкостей движения человека, множество подходов к решению этой задачи было предложено за последние десятилетия, многие из которых в последствии были применены в разработке видео игр.

Существующие подходы можно разделить на две группы: основанные на данных \cite{ArikanF}, \cite{AkikanFO}, \cite{KovarGP} и основанные на физическом моделировании \cite{AbeSP}, \cite{MacchiettoZS}, \cite{YinLP}. Такое разделение позволяет выделить ключевые идеи лежащие в основе многих из них, но не является исчерпывающим, поскольку существуют, например, гибридные подходы.

Подходы, составляющие первую группу, для создания анимаций персонажа используют большое количество предварительно записанных движений человека. Интенсивное развитие таких подходов началось после появления технологий оцифровки движений реальных объектов, таких как Motion Capture, позволивших существенно упростить сбор необходимых данных. Один из интересных примеров в этой группе -- это Motion Matching \cite{MotionMatching}. В нем анимация персонажа выбирается среди всех доступных каждый кадр. Критерии выбора могут быть разные -- вид движения, стилистика, направление. В сочетании с большой анимационной базой Motion Matching демонстрирует отличные результаты. Такие результаты обусловлены использованием данных, полученных из реального мира. Тем не менее это имеет и свои недостатки. В случае когда требуемое движение не стандартно или не может быть записано подходы основанные на данных плохо применимы.

Подходы, составляющие вторую группу, приводят персонажа в движение через управление динамикой его физической модели. В таком случае обеспечивается физическая корректность движения. Кроме того, при дополнительном моделировании окружающей персонажа среды появляется возможность обрабатывать взаимодействия с ней. К сожалению, из-за сложности разработки правил управления создание таких подходов является проблемой, также как и их использование, которое требует больших вычислительных ресурсов. Однако с углублением понимания механики движение человека и ростом производительности процессоров значимость этих недостатков будет уменьшаться. Поэтому генерация движения персонажа на основе физического моделирования остается перспективным направлением исследования.

Распространенная задача, возникающая при разработке систем управления персонажа, -- это необходимость поддержания баланса. Поддержание баланса подразумевает предотвращение неконтролируемого падения персонажа. Такая необходимость может возникнуть даже в самом простом случае, когда персонаж просто следует опорной анимации. А особенно ярко она проявляется, когда персонаж находится в присутствии внешних возмущений или на неровной поверхности. Кроме того, задача сильно варьирует в зависимости от вида воспроизводимого движения. Выделяют две категории баланса: статический, для движений на месте, и динамический.

В данной работе предлагается, реализуется и анализируется система управления персонажа, которая способна одновременно поддерживать баланс и следовать опорной анимации. Поддержание баланса основано на контроле положений центра масс и центра давления. В каждый момент времени решается задача квадратичного программирования, которая оптимизирует значения целевых функций, отвечающих за контроль положений центра масс и центра давления и следование опорной анимации. После чего персонаж приводится в движение в  с результатами оптимизации. Более того, при изменение целевых функций описываемый способ может быть адаптирован для других задач, возникающих при разработке систем управления персонажем.

\subsection{Обзор литературы}

В данном разделе рассматриваются несколько работ, идеи из которых были заимствованы при разработке предлагаемой системы управления персонажем. Так же в таблице \ref{tbl:comparison_analysis} приведен их краткий сравнительный анализ.

\subsubsection{Dynamic Postural Adjustment with QP Method}

Похожая задача поддержания баласа возникает и в другой области -- робототехнике. Причем в отличие от анимаций, где целью является увеличение реалистичности и интерактивности персонажей, в робототехнике поддержание баланса робота критически важно для его успешного функционирования в реальных условиях. Из-за сходства методов физического моделирования персонажей и роботов, идеи, предложенные в области робототехники, могут быть применены в анимации, и наоборот.

Одно из первых применений контроля положений центра масс и центра давления было сделано в \cite{KudohKI}. В этой работе описывается алгоритм восстановления сбалансированного положения робота, после воздействия на него внешних возмущений. Основу алгоритма составляет вычисление обобщенных ускорений, таких чтобы центр масс возвращался в исходное положение, а центр давления, находился внутри допустимой области. Вычисление осуществляется применением одного из двух способов: оптимизации или пропорционально-дифференцирующего регулятора. Способ выбирается в зависимости от удаленности положения центра масс от исходного. Алгоритм демонстрирует движения робота схожие с теми, что выполняет человек для восстановления баланса. В дальнейшем алгоритм был доработан в \cite{MacchiettoZS} уже применительно к анимации.

\subsubsection{Multiobjective Control with Frictional Contacts}

В \cite{AbeSP} описывается система управления персонажем способная одновременно обеспечивать контроль положения центра масс и следование персонажа опорной анимации. В таком случае основную сложность составляет одновременная работа с несколькими целями движения, поскольку опорная анимация часто не является сбалансированной. Возможность учитывать несколько конфликтующих целей движения была получена применением оптимизации, целевая функция которой содержала вклад от каждой цели движения. Не смотря на то что работа фокусировалась только на контроле положения центра масс, что недостаточно для поддержания баласа во всех ситуациях, идея комбинирования целей движения открывает множество возможностей для дальнейшего улучшения.

\subsubsection{Momentum Control for Balance}

В \cite{MacchiettoZS} описывается система управления персонажем, которая контролирует положения центра масс и центра давления, а также обеспечивает следование персонажа опорной анимации. Работа с несколькими целями движения реализуется с помощью оптимизации, во время которой вычисляются оптимальные для текущей ситуации обобщенные ускорения. Полученные данные передаются в алгоритм обратной динамики, результаты которого вместе с внешними возмущениями используются алгоритмом прямой динамики для приведения персонажа в движение. Последние две стадии используются для того, чтобы интегрировать в систему внешние возмущения аналогично алгоритму, описанному в \cite{KudohKI}.

Отличительная особенностью системы, представленной в \cite{MacchiettoZS}, -- это использование импульса и момента импульса для контроля положений центра масс и центра давления. Такой способ также будет использован в данной работе.

% This is completely meaningless comparison!
\begin{table}[ht]
  \small
  \centering
  \begin{tabular}{ p{50mm} | p{20mm} | p{20mm} | p{20mm} }
    & Контроль положения ЦМ & Контроль положения ЦД & Следование опорной анимации \\
  \hline
  Dynamic Postural Adjustment with QP Method & Да & Нет & Нет \\
  \hline
  Multiobjective Control with Frictional Contacts & Да & Нет & Да \\
  \hline
  Momentum Control for Balance & Да & Да & Да \\
  \end{tabular}
  \caption{Сравнение рассмотренных работ}
  \label{tbl:comparison_analysis}
\end{table}

\subsection{Модель персонажа}

В данном разделе описывается физическая модель персонажа и формулируются уравнения, описывающие ее.

\subsubsection{Кинематическое дерево}

Кинематическое дерево -- это система из $n$ твердых тел, соединенных между собой $m$ шарнирами. Каждый из шарниров, кроме корневого, ограничивает относительное движение тел, которые он соединяет. Например, призматический шарнир оставляет только поступательное движение вдоль выбранной оси. Корневой шарнир, в свою очередь, определяет возможность системы перемещаться в пространстве и бывает двух видов: плавающий, то есть не накладывающий ограничений, и фиксирующий.

Для того чтобы приводить в движение отдельные тела, некоторые шарниры могут быть снабжены приводами, которые генерируют необходимые силы и моменты сил. В таком случае шарниры называется активными, иначе, соответственно, неактивными. Отметим, что корневой шарнир обычно остается неактивным. Таким образом за движение системы как целого отвечает сила трения. Это сохраняет физическую корректность, но сильно усложняет управление кинематическим деревом.

Работа системы управления такой моделью сводится к вычислению сил и моментов сил, которые должны генерировать приводы, чтобы получить движение удовлетворяющие заявленным требованиям, и последующему их применению, для воспроизведения движения.

В данной работе персонаж моделируется как кинематическое дерево, все шарниры которого имеют привод, причем силы и моменты сил, генерируемые в корневом шарнире, минимизируются во время оптимизации. Результаты показывают, что такая модель в большинстве ситуаций эквивалентна кинематическому дереву с неактивным корневым шарниром. На рисунках \ref{fig:visual_model} и \ref{fig:kinematic_tree} изображен пример трехмерной модели персонажа и соответствующего кинематического дерева.

\begin{figure}
  \begin{minipage}[t]{0.475\textwidth}
    \centering
    \includegraphics[scale=0.25]{visual_model.png}
    \caption{Трехмерная модель}
    \label{fig:visual_model}
  \end{minipage}
\hfill
  \begin{minipage}[t]{0.475\textwidth}
    \centering
    \includegraphics[scale=0.25]{kinematic_tree.png}
    \caption{Кинематическое дерево. Круги обозначают шарниры, а соединения между кругами -- твердые тела}
    \label{fig:kinematic_tree}
  \end{minipage}
\end{figure}

\subsubsection{Обобщенные координаты}

При работе с кинематическим дерево важную роль играет способ, выбранный для описания положения и ориентации тел в пространстве, поскольку он во многом определят простоту, устойчивость и вычислительную сложность моделирования. Основные способы -- это максимальные координаты и обобщенные координаты.

Максимальные координаты описывают тела по отдельности, используя по 6 чисел на каждое, а ограничения, накладываемые шарнирами, учитывают при решении уравнения динамики. Такой способ позволяет использовать существующие системы физического моделирования, но страдает от ошибок работы с вещественными числами, которые приводят к тому, что тела открепляются друг от друга.

Обобщенные координаты, напротив, учитывают связи между телами. Например, для кинематического дерева, состоящего из двух тел, соединенных вращательным шарниром, используется 7 чисел, первые 6 из которых описывают положение и ориентацию одного из тел, а оставшееся -- угол поворота вокруг оси шарнира. Такой способ минимизирует количество используемых чисел и неявно учитывает ограничения.

В данной работе используются обобщенные координаты, скорости и ускорения. Отметим, что обобщенные скорости позволяют выразить скорость любого шарнира. Для этого необходимо умножить их на якобиан шарнира. После чего можно получить скорость любой точки твердого тела, прикрепленного к шарниру, зная ее положение.

\subsubsection{Уравнение динамики}

Уравнение, связывающее обобщенные ускорения и силы и моменты сил, генерируемые приводами, называется уравнением динамики кинематического дерева и имеет вид
\begin{equation*}
  H \ddot{q} + C(q, \dot{q}) + G(q) = u + J^{T} f, \tag{1.1}\label{eq:1.1}
\end{equation*}
где $q, \dot{q}, \ddot{q}$ -- обобщенные координаты, скорости и ускорения, $H$ -- матрица инерции, $C$ -- центробежная и кориолисова силы, $G$ -- сила тяготения, $u$ -- силы и моменты сил, генерируемые приводами, $J$ -- якобиан, и $f$ -- внешние силы. Вывод этого уравнения из принципа наименьшего действия описан в \cite{Featherstone}.

В качестве неизвестной в уравнении \ref{eq:1.1} может выступать $\ddot{q}$ или $u$. Алгоритмы, которые находят $\ddot{q}$, называются алгоритмами прямой динамики, а те, которые находят $u$, -- обратной. В данной работе используется алгоритм articulated rigid body, описанный в \cite{Featherstone}.

\subsubsection{Центроидальная матрица}

В \cite{OrinG} показана связь импульса и момента импульса кинематического дерева, выраженных в неподвижной системе отсчета, расположенной в центре масс, с обобщенными скоростями, имеющая следующий вид
\begin{equation*}
\begin{bmatrix} P\\ L \end{bmatrix} = A \dot{q}, \tag{1.2}\label{eq:1.2}
\end{equation*}
где $P$ -- импульс, $L$ -- момент импульса, а $A$ -- это центроидальная матрица.

Центроидальная матрица, как и матрицей инерции, является фундаментальной характеристикой кинематического дерева, которая зависит только от массы, инерции и значения обобщенных координат твердых тел, составляющих его.

При разделении центроидальной матрицы на две уравнение \ref{eq:1.2} принимает вид
\begin{align*}
  P &= A_{P} \dot{q}, \tag{1.3}\label{eq:1.3} \\
  L &= A_{L} \dot{q}. \tag{1.4}\label{eq:1.4}
\end{align*}

При дифференцировании уравнения \ref{eq:1.3} и \ref{eq:1.4} принимают вид
\begin{align*}
  \dot{P} &= \dot{A_{P}} \dot{q} + A_{P} \ddot{q}, \tag{1.5}\label{eq:1.5} \\
  \dot{L} &= \dot{A_{L}} \dot{q} + A_{P} \ddot{q}. \tag{1.6}\label{eq:1.6}
\end{align*}

Полученные уравнения \ref{eq:1.5} и \ref{eq:1.6} отражают связь между значениями производных импульса и момента импульса и обобщенными ускорениями. В данной работе они используются при формулировании целевой функции оптимизации.

\subsubsection{Импульс и момент импульса}

Положения импульса и момента импульса показывают устойчивость персонажа, поэтому одна из задач разрабатываемой системы управления -- это их контроль. Способ используемый в данной работе управляет импульсом и моментом импульса, таким образом контролируя положения центра масс и центра давления. В данном подразделе показана связь между этими величинами, которая обосновывает корректность выбранного способа.

Напомним, что центр давления -- это точка, где можно приложить результирующую сил нормальной реакции опоры и сил трения (далее -- силу реакции опоры), так чтобы момент относительно центра масс не изменился.

\begin{figure}[ht]
  \begin{minipage}{\textwidth}
    \centering
    \begin{tikzpicture}
      % character
      \node [rectangle, minimum width = 0.7cm, minimum height = 3.0cm, draw, fill = white, rotate around = {19:(1.4, -2.1)}] at (1.4, -2.1) {};
      \node [rectangle, minimum width = 0.7cm, minimum height = 2.6cm, draw, fill = white] at (-0.4, -2.7) {};
      \node [rectangle, minimum width = 0.7cm, minimum height = 2.4cm, draw, fill = white, fill opacity = 1.0] at (-0.4, -5.2) {};
      \node [rectangle, minimum width = 0.7cm, minimum height = 2.4cm, draw, fill = white, fill opacity = 1.0] at (1.26, -5.2) {};
      \node [rectangle, minimum width = 1.5cm, minimum height = 3.0cm, draw, fill = white, fill opacity = 1.0] at (0, 0) {};
      \node [circle, minimum width = 1.4cm, minimum height = 1.4cm, draw, fill = white, fill opacity = 1.0] at (0, 2.2) {};
      % ground
      \draw (-2, -6.4) -- (2.8, -6.4);

      % gravity
      \filldraw [fill = black] (0.1, -0.5) circle (2pt) node [above left] {$c$};
      \draw [-{Latex[length=4mm, width=2mm]}, line width = 0.5mm] (0.1, -0.5)  -- node[left] {$mg$} (0.1, -3.0);

      % ground reaction force
      \filldraw [fill = black] (0.2, -6.4) circle (2pt) node [below right] {$p$};
      \draw [-{Latex[length=4mm, width=2mm]}, line width = 0.5mm] (0.2, -6.4) -- node[right] {$f$} (0.8, -4.4);

    \end{tikzpicture}
    \caption{Силы}
    \label{fig:forces}
  \end{minipage}
\end{figure}

Рассмотрим силы, действующие на кинематическое дерево (рисунок \ref{fig:forces}). Запишем второй закон Ньютона и основное уравнение вращательной динамики
\begin{align*}
\dot{P} &= mg + f, \tag{1.7}\label{eq:1.7} \\
\dot{L} &= (p - c) \times f, \tag{1.8}\label{eq:1.8}
\end{align*}
где $c$ -- центр масс, $p$ -- центр давления, $m$ -- суммарная масса, и $f$ -- результирующая сил нормальной реакции опоры и сил трения.

Исключая $f$ из уравнений \ref{eq:1.7} и \ref{eq:1.8}, получим
\begin{equation*}
\dot{L} = (p - c) \times (\dot{P} - mg). \tag{1.9}\label{eq:1.9}
\end{equation*}

Полученное уравнение \ref{eq:1.9} показывает, что при известном импульсе и положении центра масс, контроль положения центра давления может быть сделан с помощью управления моментом импульса.

Теперь, запишем определение импульса кинематического дерева, рассматривая его как систему твердых тел
\begin{equation*}
  P = \sum_{i = 1}^{n} m_{i} \dot{x_{i}},
\end{equation*}
где $x_{i}$, $m_{i}$ -- положения и массы. Используя следующую цепочку равенств
\begin{equation*}
  \sum_{i = 1}^{n} m_{i} \dot{x_{i}} = \frac{d}{dt} (\sum m_{i} x_{i}) = \frac{d}{dt} (mc) = m \dot{c},
\end{equation*}
преобразуем уравнение к виду
\begin{equation}
  P = m \dot{c}, \tag{1.10}\label{eq:1.10}
\end{equation}
а дифференцированием к виду
\begin{equation}
  \dot{P} = m \ddot{c}. \tag{1.11}\label{eq:1.11}
\end{equation}

Полученное уравнения \ref{eq:1.11} показывает, что контроль ускорения центра масс может быть сделан с помощью управления производной импульса.