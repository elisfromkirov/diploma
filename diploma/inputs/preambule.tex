\usepackage{fontspec}
\usepackage{polyglossia}
\usepackage[a4paper, lmargin=30mm, rmargin=15mm, tmargin=20mm, bmargin=20mm]{geometry}
\usepackage{multirow}

\setdefaultlanguage{russian}
\setotherlanguage{english}

\defaultfontfeatures{Ligatures=TeX}

\setmainfont{Times New Roman}
\setmonofont{Courier New}
\setsansfont{Arial}

\newfontfamily\cyrillicfont{Times New Roman}
\newfontfamily\cyrillicfontsf{Arial}
\newfontfamily\cyrillicfonttt{Courier New}

\newfontfamily\englishfont{Times New Roman}
\newfontfamily\englishfontsf{Arial}
\newfontfamily\englishfonttt{Courier New}

\linespread{1.5}

\usepackage[backend=biber,
  bibencoding=utf8,
  sorting=none,
  style=gost-numeric,
  language=autobib,
  autolang=other,
  clearlang=true,
  defernumbers=true,
  sortcites=true,
  doi=true,
  isbn=true,
  ]{biblatex}

\usepackage[dvipsnames]{xcolor}
\usepackage[
  hidelinks,
  colorlinks=true,
  linkcolor=Blue, % This is the color of table of contents and figure references
  citecolor=Blue, % This is the color of citations
  urlcolor=Blue % This is the colro of urls
]{hyperref}
\renewcommand{\UrlFont}{\small\rmfamily\tt}

\renewcommand\thesection{\arabic{section}}

\usepackage{amsfonts}
\usepackage{amsmath}

\usepackage{graphicx}
\graphicspath{ {./images/} } % This sets path to directory containing images.
\addto\captionsrussian{\renewcommand{\figurename}{Рисунок}} % This changes caption from 'Рис.' to 'Рисунок'.
