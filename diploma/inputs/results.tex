\section{Результаты}

Описанная система уравнения персонажем была реализована на языке Python. Работа с кинематическим деревом реализована с помощью библиотеки pinocchio \cite{Carpentier}, \cite{Pinocchio}. Оптимизация выполняется с помощью библиотеки cvxopt \cite{CVXOPT}. Данные библиотеки выбраны поскольку они реализуют необходимый функционал наиболее эффективным образом.

% NOTE: These values were selected experimentally
\begin{figure}
  \hfill
  \begin{minipage}{0.326\textwidth}
    \centering
    \includegraphics[scale=0.13]{animation/1.png}
  \end{minipage}
  \begin{minipage}{0.326\textwidth}
    \centering
    \includegraphics[scale=0.13]{animation/2.png}
  \end{minipage}
  \begin{minipage}{0.326\textwidth}
    \centering
    \includegraphics[scale=0.13]{animation/3.png}
  \end{minipage}
  \vfill
  \hfill
  \begin{minipage}{0.326\textwidth}
    \centering
    \includegraphics[scale=0.13]{animation/4.png}
  \end{minipage}
  \begin{minipage}{0.326\textwidth}
    \centering
    \includegraphics[scale=0.13]{animation/5.png}
  \end{minipage}
  \begin{minipage}{0.326\textwidth}
    \centering
    \includegraphics[scale=0.13]{animation/6.png}
  \end{minipage}
  \vfill
  \hfill
  \begin{minipage}{0.326\textwidth}
    \centering
    \includegraphics[scale=0.13]{animation/7.png}
  \end{minipage}
  \begin{minipage}{0.326\textwidth}
    \centering
    \includegraphics[scale=0.13]{animation/8.png}
  \end{minipage}
  \begin{minipage}{0.326\textwidth}
    \centering
    \includegraphics[scale=0.13]{animation/9.png}
  \end{minipage}
  \vfill
  \hfill
  \begin{minipage}{0.326\textwidth}
    \centering
    \includegraphics[scale=0.13]{animation/10.png}
  \end{minipage}
  \begin{minipage}{0.326\textwidth}
    \centering
    \includegraphics[scale=0.13]{animation/11.png}
  \end{minipage}
  \begin{minipage}{0.326\textwidth}
    \centering
    \includegraphics[scale=0.13]{animation/12.png}
  \end{minipage}
  \vfill
  \hfill
  \begin{minipage}{0.326\textwidth}
    \centering
    \includegraphics[scale=0.13]{animation/13.png}
  \end{minipage}
  \begin{minipage}{0.326\textwidth}
    \centering
    \includegraphics[scale=0.13]{animation/14.png}
  \end{minipage}
  \begin{minipage}{0.326\textwidth}
    \centering
    \includegraphics[scale=0.13]{animation/15.png}
  \end{minipage}
  \vfill
  \hfill
  \begin{minipage}{0.326\textwidth}
    \centering
    \includegraphics[scale=0.13]{animation/16.png}
  \end{minipage}
  \begin{minipage}{0.326\textwidth}
    \centering
    \includegraphics[scale=0.13]{animation/17.png}
  \end{minipage}
  \begin{minipage}{0.326\textwidth}
    \centering
    \includegraphics[scale=0.13]{animation/18.png}
  \end{minipage}
  \caption{Опорная анимация}
  \label{fig:reference}
\end{figure}

% NOTE: These values were selected experimentally
\begin{figure}
  \hfill
  \begin{minipage}{0.325\textwidth}
    \centering
    \includegraphics[scale=0.13]{balance/1.png}
  \end{minipage}
  \begin{minipage}{0.325\textwidth}
    \centering
    \includegraphics[scale=0.13]{balance/2.png}
  \end{minipage}
  \begin{minipage}{0.325\textwidth}
    \centering
    \includegraphics[scale=0.13]{balance/3.png}
  \end{minipage}
  \vfill
  \hfill
  \begin{minipage}{0.325\textwidth}
    \centering
    \includegraphics[scale=0.13]{balance/4.png}
  \end{minipage}
  \begin{minipage}{0.325\textwidth}
    \centering
    \includegraphics[scale=0.13]{balance/5.png}
  \end{minipage}
  \begin{minipage}{0.325\textwidth}
    \centering
    \includegraphics[scale=0.13]{balance/6.png}
  \end{minipage}
  \vfill
  \hfill
  \begin{minipage}{0.325\textwidth}
    \centering
    \includegraphics[scale=0.13]{balance/7.png}
  \end{minipage}
  \begin{minipage}{0.325\textwidth}
    \centering
    \includegraphics[scale=0.13]{balance/8.png}
  \end{minipage}
  \begin{minipage}{0.325\textwidth}
    \centering
    \includegraphics[scale=0.13]{balance/9.png}
  \end{minipage}
  \vfill
  \hfill
  \begin{minipage}{0.325\textwidth}
    \centering
    \includegraphics[scale=0.13]{balance/10.png}
  \end{minipage}
  \begin{minipage}{0.325\textwidth}
    \centering
    \includegraphics[scale=0.13]{balance/11.png}
  \end{minipage}
  \begin{minipage}{0.325\textwidth}
    \centering
    \includegraphics[scale=0.13]{balance/12.png}
  \end{minipage}
  \vfill
  \hfill
  \begin{minipage}{0.325\textwidth}
    \centering
    \includegraphics[scale=0.13]{balance/13.png}
  \end{minipage}
  \begin{minipage}{0.325\textwidth}
    \centering
    \includegraphics[scale=0.13]{balance/14.png}
  \end{minipage}
  \begin{minipage}{0.325\textwidth}
    \centering
    \includegraphics[scale=0.13]{balance/15.png}
  \end{minipage}
  \vfill
  \hfill
  \begin{minipage}{0.325\textwidth}
    \centering
    \includegraphics[scale=0.13]{balance/16.png}
  \end{minipage}
  \begin{minipage}{0.325\textwidth}
    \centering
    \includegraphics[scale=0.13]{balance/17.png}
  \end{minipage}
  \begin{minipage}{0.325\textwidth}
    \centering
    \includegraphics[scale=0.13]{balance/18.png}
  \end{minipage}
  \caption{Сбалансированная анимация наклона}
  \label{fig:result}
\end{figure}

Модель персонажа, выбранная для демонстрации работы системы, имеет в общей сложности 38 степеней свободы, образуемых плавающим корневым шарниром и 32 вращательными шарнирами.

Также для демонстрации работы системы была разработана анимация наклона, кадры которой представлены на рисунке \ref{fig:reference}. Данная анимация не является сбалансированной, так как, например, проекция положения центр масс длительное время находится за пределами опорного полигона. Такая особенность добавлена для более наглядной демонстрации результатов работы системы.

Кадры движения персонажа, полученного в результате работы системы, представлены на рисунке \ref{fig:result}. Анимация наклона описанная ранее была использована в качестве опорной. Основной цикл системы выполнялся 120 раз в секунду.

% Стоит обратить внимание на следующие кадры, которые демонстрируют успешное применение системы.

% Стоит обратить особое внимание на кадры, отдельно вынесенные на рисунок \ref{fig:details}. На них видно, как в середине наклона, персонаж отклоняет таз назад, для того чтобы сохранить баланс.

% \begin{figure}[h]
%   \centering
%   \begin{minipage}{0.325\textwidth}
%     \centering
%     \includegraphics[scale=0.13]{balance/9.png}
%   \end{minipage}
%   \begin{minipage}{0.325\textwidth}
%     \centering
%     \includegraphics[scale=0.13]{balance/10.png}
%   \end{minipage}
%   \hfill
%   \caption{}
%   \label{fig:details}
% \end{figure}

% На

% Детальное отличие



% Подобные особенности движения можно наблюдать в реальном мире.

% Отметим, что система обладает возможностью тонкой настройки, при использовании.
% Однако если это не нужно -- то, благодаря формулировке задачи оптимизации система начнет работать без особого тюнинга
